\section{Related Work}
\label{sec:relwork}

\subsection{Software Fault Isolation}

The core of our system is based on \emph{software fault isolation}, an
idea first proposed by Wahbe et al., who developed the idea of placing
constraints on binary code in order to isolate each component into a
\emph{logical fault domain}, with isolation enforced by software rather
than hardware protection. Although this line of research initially
focused on RISC architectures with fixed-length instructions, other
researchers~\cite{Small:1997,Erlingsson:2000} extended this work to the
variable-length x86 instruction set by adding runtime checks to ensure
that control flow transfers go to a predefined set of targets. McCamant
et and Morrisett~\cite{McCamant:2006} introduced key techniques (such as
forced alignment of instructions with \texttt{nop} padding) that allow
reliable disassembly of x86 binaries, making them amenable to
verification. Around the same time, Ford and Cox~\cite{Ford:2008}
demonstrated a different approach to sandboxing native code. Their
system, VX32, used dynamic binary translation combined with the segment
registers available on 32-bit x86 to isolate code with minimal slowdown
(and, in some cases, a performance \emph{improvement} due to improved
code cache locality).

This somewhat crowded field was joined by Google's Native Client
(NaCl)~\cite{Yee:2009}. Native Client combines ideas from multiple
previous systems, such as the use of reliable disassembly and
segmentation-based memory isolation, and adds a trusted runtime in the
same process to provide a secure monitor for external calls; a
springboard/trampoline system provides secure control flow transfer to
and from the trusted runtime. Because the 64-bit x86 architecture
dropped support for hardware memory segmentation (and other
architectures, such as ARM, never supported it in the first place), the
Native Client team later developed several new
techniques~\cite{Sehr:2010} that allow NaCl to be efficiently
implemented on x86-64 and ARM. Native Client is included in the Google
Chrome web browser in order to allow untrusted web sites to run native
code safely in the browser (in contrast to earlier systems such as
Microsoft's ActiveX, where vulnerabilities in components would lead to
the compromise of entire browser). This means that the NaCl codebase is
more mature and ``battle-tested'' than most research systems, which
makes it an attractive base on which to implement our own system.

\subsection{Language-Based Isolation}

Earlier isolation based on high-level languages: SPIN (Modula-3)~\cite{Bershad:1995}

Singularity OS, specifically \cite{Hunt:2007} and \cite{Hunt:2007a} and \cite{Aiken:2006}

\subsection{Improving IPC/RPC Performance}

Lots and lots of stuff here. SFI paper cites a lot but there has been
tons of interest due to its relevance to microkernels.
